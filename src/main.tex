%---------------------------------------
%	Dokumentenklasse
%---------------------------------------

\documentclass[
    nenglish,
    toc=flat,
    toc=chapterentrywithdots,
    captions=tableabove,
    listof=entryprefix,
    fontsize=12pt,
    listof=leveldown,
    numbers=noenddot,
    bibliography=totoc, % Bindet Literaturverzeichnis in Inhaltsverzeichnis ein
    openany] % "openany" verhindert Kapitelstart auf ungeraden Seiten
{scrreprt}

%---------------------------------------
%	Laden von Paketen
%---------------------------------------

% Allgemeine hilfreiche Bibliotheken
\usepackage{lmodern}
\usepackage[T1]{fontenc}
\usepackage{ragged2e}
\usepackage{float}
\usepackage{adjustbox}
\usepackage{enumitem}
\usepackage{babel}
\usepackage{csquotes}
\usepackage{amsmath, amsfonts, amssymb}
\usepackage{graphicx}
\usepackage{xcolor}
\usepackage[letterspace=150]{microtype}
\usepackage[onehalfspacing]{setspace}
\usepackage{blindtext} % \blindtext zum evaluieren des Layouts einfügen
\usepackage{tablefootnote}
\usepackage{comment}
\usepackage{listings}

% Geoemetry %
\usepackage{geometry}
\geometry{
    top = 25mm,
    headsep = 7mm,
    left = 24mm,
    right = 24mm,
    bottom = 25mm,
%bindingoffset=8mm,
}

%Schusterjunge und Hurenkind vermeiden
\clubpenalty=10000
\widowpenalty=10000
\displaywidowpenalty=10000

% Caption %
\usepackage[labelfont={bf,sf},font={sf}, labelsep=space, singlelinecheck=off]{caption}

\captionsetup[figure]{justification=raggedright,format=plain, size=footnotesize}
\captionsetup[table]{justification=raggedright, size=footnotesize}

\addto\captionsngerman{\renewcommand{\figurename}{Abb.}}  % Überschreibt "Abbildung" mit Abb.
\addto\captionsngerman{\renewcommand{\tablename}{Tab.}}   %Überschreibt "Tabelle" mit Tab.

%Tabellen
%\usepackage{booktabs}
\renewcommand{\arraystretch}{1.25} %Zeilenhöhe


% Hyperref %
\usepackage{hyperref}
\hypersetup{
    colorlinks=true,
    linkcolor=black,
    urlcolor=blue,
    citecolor=black,
}

% Einfacher Umgang mit Einheiten %
\usepackage{siunitx}
\sisetup{
    locale=DE,
    detect-all,
    per-mode=symbol,
    input-symbols={e},
    table-alignment-mode=format,
    table-number-alignment = center
}

\lstset{
    language=Python,
    basicstyle=\ttfamily,
    keywordstyle=\color{violet},
    stringstyle=\color{blue},
    commentstyle=\color{green},
    morekeywords={
        polymerdelim, chemfig, phantom, chemname, charge, schemestart, schemestop, quad, qquad, scriptscriptstyle, arrow, chemnameinit}
    ,
    morecomment=[l]{\#},
    frame=shadowbox,
    breaklines=true,
    rulesepcolor=\color{gray}
}

% Bessere Kompatibilität der Dokumentenklasse mit div. Paketen %
\usepackage{scrhack}

% Moleküle direkt im Dokument zeichnen
\usepackage[version=4,arrows=pgf]{mhchem}
\usepackage{chemfig}
\setchemfig{atom sep=2em, cram width=5pt} % Legt Bindungslängen fest


% Diagramme erstellen mit pgfplots
\usepackage{pgfplots}
\pgfplotsset{compat=1.9, width=14cm, height=11.1cm} % Überschreibt die default Abmessungen der plots
\maxdeadcycles=1000

%---------------------------------------
%	Weitere Konfigurationen
%---------------------------------------

%Kopfzeile mit Seitenzahl
\usepackage[autooneside=false]{scrlayer-scrpage}
\clearpairofpagestyles
\ohead{\pagemark} % Beispiel: Seitenzahl in der äußeren Kopfzeile
\ihead{\headmark} % Beispiel: Kapitelname in der inneren Kopfzeile
\setkomafont{pageheadfoot}{\normalsize\itshape} % Schriftart für Kopf- und Fußzeilen
\KOMAoptions{headsepline=.4pt:1\textwidth} % % Linie unter der Kopfzeile
%\KOMAoptions{footsepline=.4pt:1\textwidth}


%Konfiguration Verzeichnisse
\BeforeStartingTOC[toc]{\singlespacing}
\BeforeStartingTOC[lot]{\renewcommand\autodot{:}}
\BeforeStartingTOC[lof]{\renewcommand\autodot{:}}

%Anpassung der Kapitelüberschrift
\renewcommand*{\chapterpagestyle}{scrheadings}
\RedeclareSectionCommand[
    beforeskip=0pt,
    afterskip=12pt,
    afterindent = false,
    font=\Large]{chapter}

\RedeclareSectionCommand[%
    beforeskip=14pt, % Abstand vor \section
    afterskip=8pt,   % Abstand nach \section
    afterindent=false, % Kein Einzug nach der Überschrift
    font=\large\bfseries % Schriftstil
]{section}

\RedeclareSectionCommand[%
    beforeskip=10pt, % Abstand vor \subsection
    afterskip=6pt,   % Abstand nach \subsection
    afterindent=false, % Kein Einzug nach der Überschrift
    font=\normalsize\bfseries % Schriftstil
]{subsection}

% Verhindere vertikale Streckung um Seite vollständig auszufüllen
\raggedbottom


%\input{MeineDaten}

%%%%%%%%%%%%%%%%%%%%%%%%%%%%%%%%%%
% Konfiguration des Anhangsverzeichnis
%
% weitere Informationen: https://komascript.de/comment/5578#comment-5578, mit Anpassungen
% Beispiel: https://komascript.de/comment/5609#comment-5609
%
%%%%%%%%%%%%%%%%%%%%%%%%%%%%%%%%%%

\DeclareNewTOC[%
  owner=\jobname,
  listname={Appendix list},% Titel des Verzeichnisses
]{atoc}% Dateierweiterung (a=appendix, toc=table of contents)
\DeclareNewTOC[%
  listname={Abbildungen im Anhang},% Titel des Verzeichnisses
  name=\noexpand\listoflofentryname,
]{alof}% Dateierweiterung (a=appendix, lof=list of figures)
\DeclareNewTOC[%
  listname={Tabellen im Anhang},% Titel des Verzeichnisses
  name=\noexpand\listoflotentryname
]{alot}% Dateierweiterung (a=appendix, lot=list of tables)

\makeatletter
\AfterTOCHead[atoc]{\let\if@dynlist\if@tocleft}% Damit das Anhangsverzeichnis sections einrückt
\newcommand*{\useappendixtocs}{%
  \renewcommand*{\ext@toc}{atoc}%
  \scr@ifundefinedorrelax{hypersetup}{}{% damit es auch ohne hyperref funktioniert
    \hypersetup{bookmarkstype=atoc}%
  }%
  \renewcommand*{\ext@figure}{alof}%
  \renewcommand*{\ext@table}{alot}%
}
\newcommand*{\usestandardtocs}{%
  \renewcommand*{\ext@toc}{toc}%
  \scr@ifundefinedorrelax{hypersetup}{}{% damit es auch ohne hyperref funktioniert
    \hypersetup{bookmarkstype=toc}%
  }%
  \renewcommand*{\ext@figure}{lof}%
  \renewcommand*{\ext@table}{lot}%
}
\scr@ifundefinedorrelax{ext@toc}{%
  \newcommand*{\ext@toc}{toc}
  \renewcommand{\addtocentrydefault}[3]{%
    \expandafter\tocbasic@addxcontentsline\expandafter{\ext@toc}{#1}{#2}{#3}%
  }
}{}
\makeatother

\usepackage{xpatch}
\xapptocmd\appendix{%
  \addpart{\appendixname}
  \useappendixtocs
  \listofatocs

}{}{}

\BeforeStartingTOC[atoc]{\singlespacing}
\BeforeStartingTOC[alot]{\singlespacing\renewcommand\autodot{:}}
\BeforeStartingTOC[alof]{\singlespacing\renewcommand\autodot{:}}


%Ebenen im Inhaltsverzeichnis
\newcommand{\nocontentsline}[3]{}
\newcommand{\tocless}[2]{\bgroup\let\addcontentsline=\nocontentsline#1{#2}\egroup}
\KOMAoptions{toc=indented}

%Verwendung normaler "Gänsefüßchen"
\MakeOuterQuote{"}

%Kein Einzug nach Absatz
\setlength\parindent{0pt}

%-------------------------------------------------
%	Beginn des Dokuments und Einbinden der Kapitel
%-------------------------------------------------

\begin{document}

    %\input{Konfigurationsdateien/Titelseite}

    %\input{Konfigurationsdateien/Erklaerung}

% Danksagung, Einbinden der auzufüllenden Danksagung durch einkommentieren:

    %\input{Konfigurationsdateien/Danksagung}

    \pagenumbering{Roman}
    \tableofcontents
    %\input{Konfigurationsdateien/Abbildungsverzeichnis}

%%%%%%Nutzung des Abkürzungs- und Formelzeichenverzeichnis%%%%%%%

%Für Nutzung des Abkürzungsverzeichnis folgendes einkommentieren:

    %\input{Formelzeichen und Abkuerzungen/Abkuerzungen}

%Für Nutzung des Formelverzeichnis folgendes einkommentieren:

    %\input{Formelzeichen und Abkuerzungen/Formelzeichen}

%%%%%%%%%%%%%%%%%%%%%%%%%%%%%%%%%%%%%%%%%%%%%%%%%%%%%%%%%%%%%%%%%

    \cleardoublepage
    \pagenumbering{arabic}

    \chapter{Alcanes}\label{sec:alcanes}


\section{Butane [\ref{app:butane}]}\label{sec:butane}
\chemfig{H-C(-[2]H)(-[-2]H)-C(-[2]H)(-[-2]H)-C(-[2]H)(-[-2]H)-C(-[2]H)(-[-2]H)-H}
    \chapter{Alcohols}\label{sec:alcohols}
\section{Butanol [\ref{app:butanol}]}\label{sec:butanol}
\chemfig{H-C(-[2]H)(-[-2]H)-C(-[2]H)(-[-2]H)-C(-[2]H)(-[-2]H)-C(-[2]H)(-[-2]H)-OH}

\section{Cyclohexanol [\ref{app:cyclohexanol}]}\label{sec:cyclohexanol}
\chemfig{*6(----(-OH)--)}

\section{Hexanol [\ref{app:hexanol}]}\label{sec:hexanol}
\chemfig[angle increment=30]{-[-1]-[1]-[-1]-[1]-[-1](-[1]OH)}

\section{Isobutanol [\ref{app:isobutanol}]}\label{sec:isobutanol}
\chemfig[angle increment=30]{H_3C-[1]C(<:[-2.75]H)(-[3]OH)<[-1]CH_3}

\section{Methanol [\ref{app:methanol}]}\label{sec:methanol}
\chemfig{H-C(-[2]H)(-[-2]H)-OH}

\section{Mannitol [\ref{app:mannitol}]}\label{sec:mannitol}
\chemfig[angle increment=30]{HO-[1]-[-1](<:[-3]OH)-[1](<[3]OH)-[-1](<[-3]OH)-[1](<:[3]OH)-[-1]-[1]OH}
    \chapter{Carboxylic Acids}\label{sec:carboxylic-acids}


\section{Acetic Acid [\ref{app:acetic-acid}]}\label{sec:acetic-acid}
\chemfig[]{H-C(-[2]H)(-[-2]H)-C(-[-1]OH)=[1]O}


\section{Acrylic Acid [\ref{app:acrylic-acid}]}\label{sec:acrylic-acid}
\chemfig[angle increment=30]{=^[-1]-[1](-[-1]OH)=[3,0.8]O}


\section{Benzoic Acid [\ref{app:benzoic-acid}]}\label{sec:benzoic-acid}
\chemfig{*6(-=-=(-(=[3.333]O)(-[0.666]OH))-=)}


\section{Fumaric Acid [\ref{app:fumaric-acid}]}\label{sec:fumaric-acid}
\chemfig{HO-[:-30](-[-2](=_[:-30](-[-2](=[:-30]O)(-[4.666]HO))))=[:30]O}


\section{Maleic Acid [\ref{app:maleic-acid}]}\label{sec:maleic-acid}
\chemfig[baseline=(b.base)]{HO-[-0.66](=[0.666]O)(*6(-@{b}=-(-[2]OH)(=[-0.666]O)))}


\section{Oleic Acid [\ref{app:oleic-acid}]}\label{sec:oleic-acid}
\chemfig[]
{HO-[0.666](=[2,0.8]O)-[-0.666]-[0.666]-[-0.666]-[0.666]-[-0.666]-[0.666]-[-0.666]-[0.666]=_-[-0.666]-[0.666]-[-0.666]-[0.666]-[-0.666]-[
    0.666]-[-0.666]-[0.666]}
    \chapter{Carbohydrates}\label{sec:carbohydrates}
\section{Glucose [\ref{app:glucose}]}\label{sec:glucose}
\chemfig{(-[0.666]H)(=[3.333]O)-[-2](-OH)(-[4]H)-[-2](-[4]HO)(-H)-[-2](-OH)(-[4]H)-[-2](-OH)(-[4]H)-[-2]CH_2OH}

\section{Mannose [\ref{app:mannose}]}\label{sec:mannose}
\chemfig{(-[0.666]H)(=[3.333]O)-[-2](-[4]HO)(-H)-[-2](-[4]HO)(-H)-[-2](-OH)(-[4]H)-[-2](-OH)(-[4]H)-[-2]CH_2OH}

\section{Galactose [\ref{app:galactose}]}\label{sec:galactose}
\chemfig{(-[0.666]H)(=[3.333]O)-[-2](-OH)(-[4]H)-[-2](-[4]HO)(-H)-[-2](-[4]HO)(-[0]H)-[-2](-OH)(-[4]H)-[-2]CH_2OH}

\section{Ribose [\ref{app:ribose}]}\label{sec:ribose}
\chemfig{(-[0.666]H)(=[3.333]O)-[-2](-OH)(-[4]H)-[-2](-OH)(-[4]H)-[-2](-OH)(-[4]H)-[-2]CH_2OH}

\section{Xylose [\ref{app:xylose}]}\label{sec:xylose}
\chemfig{(-[0.666]H)(=[3.333]O)-[-2](-OH)(-[4]H)-[-2](-[4]HO)(-[0]H)-[-2](-[0]OH)(-[4]H)-[-2]CH_2OH}

\section{Fructose [\ref{app:fructose}]}\label{sec:fructose}
\chemfig{CH_2OH-[-2](=O)-[-2](-[4]HO)(-H)-[-2](-OH)(-[4]H)-[-2](-OH)(-[4]H)-[-2]CH_2OH}


    \section{Polymers}\label{sec:Polymers}

\subsection{Poly(Bisphenol A Carbonate)}\label{sec:Poly(Bisphenol A Carbonate)}
\begin{tiny}
    \chemfig{\phantom{-}@{op}(-[0.666](=[2]O)(-[-0.666]O(-[0.666]*6(=-=(-[0.666](-[2.666]H_3C)(-[1.333]CH_3)(-[-0.666](*6(=-=(-[-0.666]O(-[0.666]@{cl}))-=-))))-=-))))}
\polymerdelim[open xshift = 7.5pt, close xshift = 3.5pt, height = 45pt, depth = 10pt, delimiters={[]}, indice = \!\!n]{op}{cl}
\end{tiny}


\subsection{Polyurethane}\label{sec:Polyurethane}
\begin{tiny}
    \chemfig[]{-[@{op,.5}]C(=[2]O)-N(-[-2]H)-*6(-=-(-C(-*6(=-=(-N(-[-2]H)-C(=[2]O)-O-C(-[2]H)(-[-2]H)-C(-[2]H)(-[-2]H)-O-[@{cl,0.5}])-=-))(-[2]H)(-[-2]h))=-=)}
\polymerdelim[height = 20pt, depth = 20pt, delimiters={[]}, indice = \!\!n]{op}{cl}
\end{tiny}
    \chapter{Mass Spectrometry}\label{sec:Mass Spectrometry}


\section{1-Propanol [\ref{app:1-propanol}]}\label{sec:1-Propanol}
\begin{figure}[H]
{\footnotesize
\schemestart
\chemname{\chemfig{H-C(-[2]H)(-[-2]H)-C(-[2]H)(-[-2]H)-C(-[2]H)(-[-2]H)-\charge{90:1pt=\|,-90:1pt=\|}{O}-H}}{Propan-1-ol}
\+
\chemfig{\charge{45=$\scriptscriptstyle{-}$}{e}}
\arrow(.mid east--.mid west)
\chemname
{\chemfig{H-C(-[2]H)(-[-2]H)-C(-[2]H)(-[-2]H)-C(-[2]H)(-[-2]H)-\charge{115:1pt=\.,-90:1pt=\|,60:1pt=$\scriptscriptstyle{+}$}{O}-H}}
{m/z\,=\,60}
\+
\chemfig{2\,\charge{45=$\scriptscriptstyle{-}$}{e}}
\arrow(@c2--n1)[-70,1.5]
\chemname{\chemfig{\charge{180:1pt=$\scriptscriptstyle{+}$}{C}(-[2]H)(-[-2]H)-\charge{90:1pt=\|,-90:1pt=\|}{O}-H}}{\phantom{XXX}}
\arrow(.mid east--.mid west)[0,0.7]
\chemnameinit{}
\chemname{\chemfig{C(-[3]H)(-[-3]H)=\charge{90:2pt=$\scriptscriptstyle{+}$,-90:1pt=\|}{O}-H}}{m/z\,=\,31}
\chemnameinit{}
\arrow(@c2--n2)[-100,5] % länge ggf variieren
\chemname{\chemfig{H-C(-[2]H)(-[-2]H)-\charge{90:2pt=$\scriptscriptstyle{+}$}{C}(-[-2]H)-\charge{0:1pt=\.}{C}(-[2]H)(-[-2]H)}}
{m/z\,=\,42}\qquad
\+\qquad
\chemname{\chemfig{H_2O}}{$\Delta$m\,=\,18}
\chemnameinit{}
\arrow(@c2--n4)[225,3]
\chemname
{\chemfig{H-C(-[2]H)(-[-2]H)-C(-[2]H)(-[-2]H)-C(-[2]H)(-[-2]H)-\charge{90:1pt=\|,-90:1pt=\|,0:2pt=$\scriptscriptstyle{+}$}{O}}}
{m/z\,=\,59}\qquad
\+
\chemname{\chemfig{\charge{0:1pt=\.}{H}}}{$\Delta$m\,=\,1}
\arrow(@n1--nn1)[-90,0.45,white]
\chemname{\chemfig{H-C(-[2]H)(-[-2]H)-\charge{0:1pt=\.}{C}(-[2]H)(-[-2]H)}}{$\Delta$m\,=\,29}
\schemestop
}
\end{figure}

\section{Ethylbenzene [\ref{app:ethylbenzene}]}\label{sec:ethylbenzene}
\begin{figure}[H]
{\scriptsize
    \schemestart
    \chemname{\chemfig{*6((-H)-(-H)=(-H)-(-H)=(-[2](-[0.666]CH_3))-(-H)=(-H))}}{Ethylbenzol} \qquad
    \+ \qquad
    \chemfig{\charge{45=$\scriptscriptstyle{-}$}{e}}
    \arrow(.mid east--.mid west)
    \chemname{\chemfig{*6(\charge{30:3pt=\.}{}(-H)-(-H)=(-H)-(-H)=(-[2](-[0.666]CH_3))-\charge{-30:3pt=$\scriptscriptstyle{+}$}{}(-H)-(-H))}}{m/z\,=\,106} \qquad
    \+ \qquad
    \chemfig{2\,\charge{45=$\scriptscriptstyle{-}$}{e}}
    \arrow(@c2--n1)[-60,1.5]
    \chemname{\chemfig{**[0,360,dash pattern=on 2pt off 2pt]7(\charge{25:18.5pt=\+}{}(-H)-(-H)-(-H)-(-H)-(-H)-(-H)-(-H)-(-H))}}{m/z\,=\,91}
    \qquad \+ \qquad % +
    \chemname{\chemfig{\charge{180:2pt=\.}{C}H_3}}{$\Delta$m\,=\,15}
    \arrow(@n1--m2)[-90,1]
    \chemname{\chemfig{**[0,360,dash pattern=on 2pt off 2pt]5(\charge{35:13pt=\+}{}(-H)-(-H)-(-H)-(-H)-(-H)-(-H)-(-H)-)}}{m/z\,=\,65}
    \qquad \+ \qquad
    \chemname{\chemfig{C_2H_2}}{$\Delta$m\,=\,26}
    \arrow(@c2--n4)[225,2.5]
    \chemname{\chemfig{*6((-H)-(-H)=(-H)-(-H)=\charge{90:3pt=$\scriptscriptstyle{+}$}{}-(-H)=(-H))}}{m/z\,=\,77}
    \qquad \+ \qquad
    \chemname{\chemfig{\charge{180:2pt=\.}{C}_2H_5}}{$\Delta$m\,=\,29}
    \arrow(@n4--m2)[-90,1]
    \chemname{\chemfig{*4((-H)-(-H)=(-H)-\charge{115:3pt=$\scriptscriptstyle{+}$}{}=)}}{m/z\,=\,51}
    \qquad \+ \qquad
    \chemname{\chemfig{C_2H_2}}{$\Delta$m\,=\,26}
\schemestop
}
\end{figure}


    \appendix %Entfernen/Auskommentieren um Anhang zu deaktivieren
    \chapter{Carboxylic Acids}\label{app:carboxylic-acids}


\section{Acetic Acid}\label{app:acetic-acid}
\begin{lstlisting}
\chemfig[]{H-C(-[2]H)(-[-2]H)-C(-[-1]OH)=[1]O}
\end{lstlisting}


\section{Acrylic Acid}\label{app:acrylic-acid}
\begin{lstlisting}
\chemfig[angle increment=30]{=^[-1]-[1](-[-1]OH)=[3,0.8]O}
\end{lstlisting}


\section{Benzoic Acid}\label{app:benzoic-acid}
\begin{lstlisting}
\chemfig{*6(-=-=(-(=[3.333]O)(-[0.666]OH))-=)}
\end{lstlisting}


\section{Fumaric Acid}\label{app:fumaric-acid}
\begin{lstlisting}
\chemfig{HO-[:-30](-[-2](=_[:-30](-[-2](=[:-30]O)(-[4.666]HO))))=[:30]O}
\end{lstlisting}


\section{Maleic Acid}\label{app:maleic-acid}
\begin{lstlisting}
\chemfig[baseline=(b.base)]{HO-[-0.66](=[0.666]O)(*6(-@{b}=-(-[2]OH)(=[-0.666]O)))}
\end{lstlisting}


\section{Oleic Acid}\label{app:oleic-acid}
\begin{lstlisting}
\chemfig[]
{HO-[0.666](=[2,0.8]O)-[-0.666]-[0.666]-[-0.666]-[0.666]-[-0.666]-[0.666]-[-0.666]
-[0.666]=_-[-0.666]-[0.666]-[-0.666]-[0.666]-[-0.666]-[0.666]-[-0.666]
-[0.666]}
\end{lstlisting}


\chapter{polymers}\label{app:polymers}


\section{Poly(Bisphenol A Carbonate)}\label{app:poly(bisphenol-a-carbonate)}
\begin{lstlisting}
\chemfig
{\phantom{-}@{op}(-[0.666](=[2]O)(-[-0.666]O(-[0.666]*6(=-=(-[0.666](-[2.666]H_3C)(-[1.333]CH_3)(-[-0.666](*6(=-=(-[-0.666]O(-[
    0.666]@{cl}))-=-))))-=-))))}
    \polymerdelim[open xshift = 7.5pt, close xshift = 3.5pt, height = 45pt, depth = 10pt, delimiters={[]}, indice = \!\!n]{op}{cl}
\end{lstlisting}


\section{Polyurethane}\label{app:polyurethane}
\begin{lstlisting}
\chemfig[]
{-[@{op,.5}]C(=[2]O)-N(-[-2]H)-*6(-=-(-C(-*6(=-=(-N(-[-2]H)-C(=[2]O)-O-C(-[2]H)(-[-2]H)-C(-[2]H)(-[-2]H)-O-[@{cl,0.5}])-=-))(-[
    2]H)
    (-[-2]H))=-=)}
    \polymerdelim[height = 20pt, depth = 20pt, delimiters={[]}, indice = \!\!n]{op}{cl}
\end{lstlisting}


\chapter{Mass Spectrometry}\label{app:mass-spectrometry}


\section{1-Propanol}\label{app:1-propanol}
\begin{footnotesize}
    \begin{lstlisting}
\schemestart
\chemname{\chemfig{H-C(-[2]H)(-[-2]H)-C(-[2]H)(-[-2]H)-C(-[2]H)(-[-2]H)-\charge{90:1pt=\|,-90:1pt=\|}{O}-H}}{Propan-1-ol}
\+
\chemfig{\charge{45=$\scriptscriptstyle{-}$}{e}}
\arrow(.mid east--.mid west)
\chemname
{\chemfig{H-C(-[2]H)(-[-2]H)-C(-[2]H)(-[-2]H)-C(-[2]H)(-[-2]H)-\charge{115:1pt=\.,-90:1pt=\|,60:1pt=$\scriptscriptstyle{+}$}{O}-H}}
{m/z\,=\,60}
\+
\chemfig{2\,\charge{45=$\scriptscriptstyle{-}$}{e}}
\arrow(@c2--n1)[-70,1.5]
\chemname{\chemfig{\charge{180:1pt=$\scriptscriptstyle{+}$}{C}(-[2]H)(-[-2]H)-\charge{90:1pt=\|,-90:1pt=\|}{O}-H}}{\phant
\arrow(.mid east--.mid west)[0,0.7]
\chemnameinit{}
\chemname{\chemfig{C(-[3]H)(-[-3]H)=\charge{90:2pt=$\scriptscriptstyle{+}$,-90:1pt=\|}{O}-H}}{m/z\,=\,31}
\chemnameinit{}
\arrow(@c2--n2)[-100,5]
\chemname{\chemfig{H-C(-[2]H)(-[-2]H)-\charge{90:2pt=$\scriptscriptstyle{+}$}{C}(-[-2]H)-\charge{0:1pt=\.}{C}(-[2]H)(-[-2]H)}}
{m/z\,=\,42}\qquad
\+\qquad
\chemname{\chemfig{H_2O}}{$\Delta$m\,=\,18}
\chemnameinit{}
\arrow(@c2--n4)[225,3]
\chemname
{\chemfig{H-C(-[2]H)(-[-2]H)-C(-[2]H)(-[-2]H)-C(-[2]H)(-[-2]H)-\charge{90:1pt=\|,-90:1pt=\|,0:2pt=$\scriptscriptstyle{+}$}{O}}}
{m/z\,=\,59}\qquad
\+
\chemname{\chemfig{\charge{0:1pt=\.}{H}}}{$\Delta$m\,=\,1}
\arrow(@n1--nn1)[-90,0.45,white]
\chemname{\chemfig{H-C(-[2]H)(-[-2]H)-\charge{0:1pt=\.}{C}(-[2]H)(-[-2]H)}}{$\Delta$m\,=\,29}
\schemestop
    \end{lstlisting}
\end{footnotesize}

\newpage


\section{Ethylbenzene}\label{app:ethylbenzene}
%\begin{footnotesize}
\begin{lstlisting}
\schemestart
    \chemname{\chemfig{*6((-H)-(-H)=(-H)-(-H)=(-[2](-[0.666]CH_3))-(-H)=(-H))}}{Ethylbenzol} \qquad
    \+ \qquad
    \chemfig{\charge{45=$\scriptscriptstyle{-}$}{e}}
    \arrow(.mid east--.mid west)
    \chemname{\chemfig{*6(\charge{30:3pt=\.}{}(-H)-(-H)=(-H)-(-H)=(-[2](-[0.666]CH_3))-\charge{-30:3pt=$\scriptscriptstyle{+}$}{}(-H
)-(-H))}}{m/z\,=\,106} \qquad
    \+ \qquad
    \chemfig{2\,\charge{45=$\scriptscriptstyle{-}$}{e}}
    \arrow(@c2--n1)[-60,1.5]
    \chemname{\chemfig{**[0,360,dash pattern=on 2pt off 2pt]7(\charge{25:18.5pt=\+}{}(-H)-(-H)-(-H)-(-H)-(-H)-(-H)-(-H)-(-H))}}{m/z\,=\,91}
    \qquad \+ \qquad % +
    \chemname{\chemfig{\charge{180:2pt=\.}{C}H_3}}{$\Delta$m\,=\,15}
    \arrow(@n1--m2)[-90,1]
    \chemname{\chemfig{**[0,360,dash pattern=on 2pt off 2pt]5(\charge{35:13pt=\+}{}(-H)-(-H)-(-H)-(-H)-(-H)-(-H)-(-H)-)}}{m/z\,=\,65}
    \qquad \+ \qquad
    \chemname{\chemfig{C_2H_2}}{$\Delta$m\,=\,26}
    \arrow(@c2--n4)[225,2.5]
    \chemname{\chemfig{*6((-H)-(-H)=(-H)-(-H)=\charge{90:3pt=$\scriptscriptstyle{+}$}{}-(-H)=(-H))}}{m/z\,=\,77}
    \qquad \+ \qquad
    \chemname{\chemfig{\charge{180:2pt=\.}{C}_2H_5}}{$\Delta$m\,=\,29}
    \arrow(@n4--m2)[-90,1]
    \chemname{\chemfig{*4((-H)-(-H)=(-H)-\charge{115:3pt=$\scriptscriptstyle{+}$}{}=)}}{m/z\,=\,51}
    \qquad \+ \qquad
    \chemname{\chemfig{C_2H_2}}{$\Delta$m\,=\,26}
\schemestop
\end{lstlisting}
%\end{footnotesize}


 %Entfernen/Auskommentieren um Anhang zu deaktivieren

\end{document}